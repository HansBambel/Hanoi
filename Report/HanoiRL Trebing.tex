\documentclass[a4paper]{article}
\usepackage{amsmath}
\usepackage{multicol}

\title{Foundations of Agents: Practical Assignment 1}
\author{Kevin Trebing (i6192338)}

\begin{document}
\maketitle

\section{Tower of Hanoi problem}
In a Tower of Hanoi problem the agent needs to move disks of different sizes from one pin to another pin. Furthermore, the disks need to be in the order that a smaller disk needs to be on top of a bigger one. Only one disk can be moved at a time and only the topmost disk can be moved. In our problem we have 3 pins and two disks. The starting position is pin 1 and the smaller disk is on the bigger disk. The goal is to move the disks to pin 3.

\subsection{Description of States}
Notation:
\begin{multicols}{2}
\begin{itemize}

\item a = small disk,
\item b = big disk,
\item 1 = pin1,
\item 2 = pin2,
\item 3 = pin3,
\item ab = a is on b
\end{itemize}
\end{multicols}

\noindent
We have 12 different possible states: \\

\begin{tabular}{ c|c|c|c}
	State & \multicolumn{3}{c}{Pin} \\
	\hline
	$s_{0}$ & ab1 & 2 & 3 \\
	$s_{1}$ & 1 & ab2 & 3 \\
	$s_{2}$ & 1 & 2 & ab3 \\
	$s_{3}$ & ba1 & 2 & 3 \\
	$s_{4}$ & 1 & ba2 & 3 \\
	$s_{5}$ & 1 & 2 & ba3 \\
	$s_{6}$ & b1 & a2 & 3 \\
	$s_{7}$ & a1 & b2 & 3 \\
	$s_{8}$ & b1 & 2 & a3 \\
	$s_{9}$ & a1 & 2 & b3 \\
	$s_{10}$ & 1 & a2 & b3 \\
	$s_{11}$ & 1 & b2 & a3

\end{tabular}

\newpage
\subsection{Description of Actions}
We have 6 different actions that the agent can take. \\

\begin{tabular}{c|c}
Action & effect \\
\hline
$a_{1}$ & move a to pin1 \\
$a_{2}$ & move a to pin2 \\
$a_{3}$ & move a to pin3 \\
$b_{1}$ & move b to pin1 \\
$b_{2}$ & move b to pin2 \\
$b_{3}$ & move b to pin3
\end{tabular}


\section{Optimal policy}
The optimal policy describes for every state the best action the agent should take.

\begin{multicols}{2}
\begin{itemize}
\item $\pi(s_{0}) = a_{2}$
\item $\pi(s_{1}) = a_{1}$
\item $\pi(s_{2}) = a_{1}$
\item $\pi(s_{3}) = b_{3}$
\item $\pi(s_{4}) = b_{3}$
\item $\pi(s_{5}) = b_{2}$
\item $\pi(s_{6}) = b_{3}$
\item $\pi(s_{7}) = b_{3}$
\item $\pi(s_{8}) = a_{2}$
\item $\pi(s_{9}) = a_{3}$
\item $\pi(s_{10}) = a_{3}$
\item $\pi(s_{11}) = a_{1}$
\end{itemize}
\end{multicols}

\section{Q-values}

The q-values of of the states given the optimal policy:
\begin{multicols}{2}
\begin{itemize}
\item $u(s_0) = 75.55 $
\item $u(s_1) = 75.30 $
\item $u(s_2) = 0.00 $
\item $u(s_3) = 86.82 $
\item $u(s_4) = 86.92 $
\item $u(s_5) = 65.73 $
\item $u(s_6) = 85.95 $
\item $u(s_7) = 86.03 $
\item $u(s_8) = 74.16 $
\item $u(s_9) = 98.82 $
\item $u(s_{10}) = 98.82 $
\item $u(s_{11}) = 75.29 $
\end{itemize}
\end{multicols}

\section{Difference to value and policy iteration}
The major difference is that the results could differ a lot due to randomness. The q-values sometimes are very similar to the utility values calculated from the previous assignment.

\section{Convergence speed}
TODO

\section{Note}
The Hanoi.py file requires Python 3.6.

\end{document}
